%%%%%%%%%%%%%%%%%%%%%%%%%%%%%%%%%%%%%%%%%%%%%%%
%%%%%%%%%%%%%%%%%%%%%%%%%%%%%%%%%%%%%%%%%%%%%%%
%
%This is where one would tell \LaTeX{} how to format Theorems, Definitions, etc. and also %indicate the environment names. You need the amsthm package (loaded in the woosterthesis %class) in order for these commands to work.
%
%%%%%%%%%%%%%%%%%%%%%%%%%%%%%%%%%%%%%%%%%%%%%%%
%%%%%%%%%%%%%%%%%%%%%%%%%%%%%%%%%%%%%%%%%%%%%%%

% an example of defining your own theoremstyle
%\newtheoremstyle{break}% name
%  {\topsep}%      Space above
%  {\topsep}%      Space below
%  {\itshape}%         Body font
%  {}%         Indent amount (empty = no indent, \parindent = para indent)
%  {\bfseries}% Thm head font
%  {.}%        Punctuation after thm head
%  {\newline}%     Space after thm head: " " = normal interword space;
%        %       \newline = linebreak
%  {}%         Thm head spec (can be left empty, meaning `normal')
% Kyle Kindbom requested a theorem style where the theorem header was on a
% separate line.
% The break theorem style will put a line break after the theorem header. - JB
\newtheoremstyle{break}% name
  {\topsep}%      Space above
  {\topsep}%      Space below
  {\itshape}%     Body font
  {}%             Indent amount (empty = no indent, \parindent = para indent)
  {\bfseries}%    Theorem head font
  {.}%            Punctuation after thm head
  {\newline}%     Space after theorem head: " " = normal interword space;
            %     \newline = linebreak
  {}%             Theorem head spec (can be left empty, meaning `normal')
%%%%%%%%%%%%%%%%%%%%%%%%%%%%%%%%%%%%%%%%%%%%%%
\newtheoremstyle{scthm}{\topsep}{\topsep}{\itshape}{}{\bfseries\scshape}{}{ }{}
\newtheoremstyle{itdefn}{\topsep}{\topsep}{\itshape}{}{\bfseries}{.}{ }{}
\newtheoremstyle{scdefn}{\topsep}{\topsep}{\itshape}{}{}{}{ }{\thmname{\textbf{#1}}\thmnumber{ \textbf{#2}}\thmnote{ \scshape #3:}}
\theoremstyle{break}
\newtheorem{thm}{Theorem}[section]% create a short command for theorems and number them within sections 
\newtheorem{cor}[thm]{Corollary}% create a short command for corollaries and number with theorems
\newtheorem{lem}[thm]{Lemma}% create a short command for lemmas and number with theorems
\newtheorem{prop}[thm]{Proposition}% create a short command for propositions and number with theorems
\theoremstyle{definition}% change the theorem styling
\newtheorem{defn}{Definition}[section]% create a short command for definitions and number within sections
\newtheorem{ex}{Example}% create a short command for examples and number within sections
\newtheorem{exer}{Exercise}% create a short command for exercises and number within the document
\theoremstyle{remark}% change the theorem styling
\newtheorem{rem}{Remark}[section]% create a short command for remarks and number within sections
\renewcommand{\therem}{}% set the counter for remarks
\theoremstyle{plain}% change the theorem styling
\newtheorem{note}{Notation}[section]% create a short command for notation and number within sections
\renewcommand{\thenote}{}% set the counter for notation
\newtheorem{nts}{Note to self}% create a short command for notes to self and number within the document
\renewcommand{\thents}{}% set the counter for note to self
\newtheorem{term}{Terminology}[section]% create a short command for terminology and number within sections
\renewcommand{\theterm}{}% set the counter for terminology

\theoremstyle{itdefn}
\newtheorem{bdefn}{Definition}[section]
\newsavebox{\fmbox} 
\newenvironment{boxeddefn}[2] 
{\begin{lrbox}{\fmbox}\begin{minipage}{0.9 \linewidth }\begin{singlespace}\begin{bdefn}[{#1}]\label{#2}\vspace{0.2cm}} 
{\end{bdefn}\end{singlespace}\end{minipage}\end{lrbox}\fbox{\usebox{\fmbox}}}